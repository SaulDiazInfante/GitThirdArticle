%======================================================================================================================
%                                                 STEP 2                                                              %
%======================================================================================================================

	Now we proceed with step 2. In what follows $C$ denotes a generic positive constant independent of the step $h$, and
which value could change at each appearances.
\begin{lem}\label{lem:BoundAndConvergenceOfyh}
	Assume \Cref{ass:OSLC,ass:ajBound,ass:HypThmSingularities} hold, then there is a constant $C=C(p,T)>0$ and a sufficiently small
	step size $h$, such that for all $p>2$
	\begin{equation}\label{eqn:yh-MomentBounds}
		\m\left[
			\sup_{0\leq t \leq T}
				|y_h(t)|^p
		\right]
		\leq
			C
		\left( 
			1+\m |y_0|^p
		\right).
	\end{equation}
	Moreover
	\begin{equation}\label{eqn:yh-convergence}
	\lim_{h \to 0}
	\m\left[
	\sup_{0\leq t \leq T}
	|y(t)-y_h(t)|^2
	\right]=0.
	\end{equation}
\end{lem}
\begin{pf}
	By theorem \ref{thm:MaoCoercive} and inequality \eqref{eqn:h-MonotoneCondition}, 
	we have	 bound \eqref{eqn:yh-MomentBounds}.
	On the other hand, to prove \eqref{eqn:yh-convergence} we will use the properties of 
	$\varphi_{f_h}$ and the Higham's stopping time technique employed in \cite[Thm 2.2]{Higham2002b}. 
	Note that by relationship \eqref{eqn:VarPhiEjc}, we can rewrite the Steklov function  as
	\begin{align}\label{fo}
		\varphi_{f_h}(u) 
			&=  f^{(j)}(u) \1{E_j}(u) +\Phi_j(u) f^{(j)}(u) \1{E_j^c}(u) .
	\end{align}
	 Let $u\in \R^d$, $|u|<n$ by \Cref{l1} we have $|\Phi(u)|\leq L_{\Phi}$ and since $f \in C^1(\R^d)$, there is a 
	 positive  constant $R_n$, which depends on $n$, such that 
	\begin{align*}	
		|\varphi_{f_h}^{(j)}(u) - f^{(j)}(u)|
		&\leq
			\1{E_j^c}(u)
			|f^{(j)}(u)|
			\left|
				\Phi_j(u) - 1
			\right| \notag \\
		&\leq
			\1{E_j^c}(u)
			\left(
				L_{\Phi} + 1
			\right)
			|f(u)|	 \notag \\
		&\leq
		\1{E_j^c}(u) R_n(L_{\Phi}+1), \quad \forall u \in \R^d, \quad |u|\leq n,
	\end{align*}
	for each $j\in \{1,\dots, d\}$. At the interface betweeen $E_j$ and $E_j^c$, 
	we know by the scalar L' H\^{o}pital theorem that 
	$ \lim_{h\to 0} \Phi_j(u) = 1$ for each fixed $u\in E_j^c$.
	And if $u\in E_j$,  by \Cref{ass:HypThmSingularities} and using   
	theorems \ref{thm:Lawlor} or \ref{thm:Fine}, we have 
	\begin{equation*}
	\lim_{h \to 0} F_h^{(j)}(u)
		=
		\lim_{h \to 0}
			e^{ha_j(u)} u^{(j)} + 
		\lim_{h \to 0}
			\left( \Phi_j(u)
				\1{E_j^c}(u)
				+h \1{E_j}(u)
			\right)
			b_j(u^{(j)}) 
		= u^{(j)},
	\end{equation*}
	for each $j \in \{1, \dots , d\}$, hence
	$%\begin{equation*}
		\displaystyle
		\lim_{h\to 0} F_h(u)=u.
	$ %\end{equation*}
	In addition, $g_h$ and $g$ are locally Lipschitz. So, given $n>0$ there is  a function 
	$K_n(\cdot):(0,\infty)\to (0,\infty)$ such that
	$K_n(h)\to 0$ when $h \to 0$ and
	\begin{equation}\label{eqn:PhihGhKRhBound}
		|\varphi_{f_h}(u)-f(u)|^2 \vee |g_h(u)-g(u)|^2
		\leq K_n(h) \qquad \forall u\in \R^d, \quad |u| \leq n.
	\end{equation}
	Also, $f$, $g$ are $C^{1}$, then there is  a constant $H_n>0$ such that
	\begin{equation}\label{eqn:f-gHRBound}
		|f(u)-f(v)|^2 \vee |g(u)-g(v)|^2
		\leq H_n |u-v|^2\qquad \forall u,v \in \R^d, \quad|u|\vee |v| \leq n.
	\end{equation}
	The rest of the proof run as in \citep[Lem. 3.6]{Higham2002b}. \qed
\end{pf}