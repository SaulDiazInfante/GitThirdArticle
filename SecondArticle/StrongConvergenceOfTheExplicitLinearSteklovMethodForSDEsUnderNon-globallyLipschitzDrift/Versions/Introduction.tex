	Fundamental computational tools in applications,  as Brownian Dynamics \cite{Cruz2012} or Monte Carlo simulations 
\cite{Glasserman2004} requires simple and computational cheap numerical methods  --- 
in some cases this exclude the use of implicit schemes.
In this context, the Euler Maruyama (EM) leads because has simple algebraic structure, 
cheap computational cost and acceptable convergence rate under global Lipschitz condition, but, if the drift or 
diffusion of a SDE grows faster than something linear, then the EM approximation diverges  in mean square sense 
\cite{Hutzenthaler2009, Hutzenthaler2010, Hutzenthaler2012b}.
Moreover, Giles in \cite{Giles2008} proposes an efficient variance reduction technique based on numerical strong 
convergence: the \emph{multilevel}  Monte Carlo method. 
Since applications in Finance, Biology or Physics consider stochastic differential equations (SDE) with super-linear
grow and Locally Lipschitz coefficients, designing explicit strong convergent schemes (in $L^p$ sense)
attracts the actual stochastic numerics research. 

	Given its simple structure and low computational cost, results natural develop numerical schemes (on the
above set up) by improving the EM. In this line arises the tamed schemes
\cite{Hutzenthaler2012c, Hutzenthaler2015, Wang2011, Sabanis2013, Zong2014}, a special type of balanced method 
\cite{Tretyakov2013b}, and the stopped scheme \cite{Liu2013a}, which considers local Lipschitz drift and at most linear 
grow diffusion. Recently in \cite{Mao2015} Mao, develops the truncated Euler method and  Sabanis in \cite{Sabanis2015} 
proposes a new tamed type scheme, both considers diffusion, and drift under locally Lipschitz and super-linear grow 
conditions. All of this works  prove  convergence  of their schemes following the theory developed by Higham Stuart and 
Mao in \cite{Higham2002b} or the new approach developed by  Hutzenthaler and Jentzen in \cite{Hutzenthaler2015}.
Both theories turn the problem of prove strong convergence into provide a bound for the moments of the numerical and 
continuous solution of a underlying SDE. The approach of Hutzenthaler and Jentzen add tools based on conveniently 
rare events to the well established stopping time technique of Higham et al.

	Following the work of \citeauthor*{Higham2002b}, in this work we edevelop a new explicit method for non linear SDE 
with a specific structure, we name it the Linear Steklov method (\SM). Our approach follow the same ideas of 
\cite{Diaz-Infante2015} in order to extend the explicit scalar Steklov method to a multidimensional setting. To
 motivate it, consider the  vector It\^o stochastic differential equation of the form
\begin{equation}\label{eqn:SDE1}
	dy(t)
	 =f(y(t))dt + g(y(t))dW(t), \quad 0\leq t\leq T,
	\quad y(0)=y_0,
\end{equation}
where $(f^{(1)},\dots, f^{(d)}):\R^d \to \R^d$ is one sided Lipschitz and 
$g = (g^{(i,j)})_{i\in \{1,\dots,d\}, j\in\{1,\dots, m\}}:\R^d \to \R^{d\times m}$ is global Lipschitz. Also we assume 
that  each component function $f^{(j)}$  has the structure
$$
	f^{(j)}(x) = a_j(x) x^{(j)} + b_j (x^{(-j)}), \qquad x\in \R^d, \qquad 
	x^{(-j)} = \left( x^{(1)},\dots,x^{(j-1)},x^{(j+1)},\dots x^{(d)}\right).
$$ 
Note that Stochastic models as Lotka Volterra, Duffin - Van der Pol, Lorenz, SIR, SIS follow this form. 
We work on the standard setup, that is,  $y(t)\in \R^d$ for each $t$ and  $W(t)$ is a
$m$-dimensional standard Brownian motion on a filtered and complete probability space
$
	(
		\Omega ,\calF,(\calF_t)_{t\in[0,T]},\prob{}
	)
$,
with the filtration
$(\mathcal{F}_t)_{t\in[0,T]}$  generated by the Brownian process.


In section 2... Section 3 ...

Notation