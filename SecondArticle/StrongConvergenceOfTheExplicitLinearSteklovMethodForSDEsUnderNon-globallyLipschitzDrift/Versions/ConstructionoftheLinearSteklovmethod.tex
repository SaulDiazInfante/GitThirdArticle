
\newcommand{\BigFig}[1]{\parbox{12pt}{\Huge #1}}
\newcommand{\BigZero}{\BigFig{0}}
%
In order to construct the LS method, we assume that  for each component of the drift coefficient of \eqref{eqn:SDE1} there exist 
two locally Lipschitz functions $a_j:\mathbb{R}^{d} \to \mathbb{R}$, and
		$b_{j}:\mathbb{R}^{d-1} \to \mathbb{R}$ such that 
		\begin{equation}\label{eqn:AlternativeConstruction}
		f^{(j)}(y) = a_j (y) y^{(j)} + b_{j}(y^{(-j)}), \qquad
		y^{(-j)} = (y^{(1)}, \dots ,y^{(j-1)}, y^{(j+1)}, \dots, y^{(d)}).
		\end{equation}
The Linear Steklov approximation consists in estimating the function $f$ of 
	the SDEs \eqref{eqn:SDE1} (for simplicity $d=j=1$) by 
\begin{equation} \label{ap}
	f(y(t)) \approx 
		\varphi_{f}(y(t_{\eta_{+}(t)})) =
		\left(
			\frac{1}{y(t_{\eta_+(t)})-y(t_{\eta (t)})}
			\int \limits
_				{y(t_{\eta(t)})}^{y(t_{\eta_+(t)})}
					\frac{du}
						{
							a(y(t_{\eta(t)}))u
							+b
						}
	\right)^{-1}, \qquad t\in [0,T],
\end{equation}
where  $\varphi_f$ 
is the linearized Steklov average \cite{Diaz-Infante2015}. 
Now, using \eqref{ap} we develop the new  method as follows: 
First, let us take  $0=t_0 < t_1< \cdots < t_N=T$ a partition of the interval $[0,T]$ with 
constant step-size $h=T/N$ and such that $t_k=kh$ for $k=0,\ldots, N$.  By discretization
of the SDEs \eqref{eqn:SDE1}, we have for  each component equation $j\in\{1, \ldots,  d\}$ 
	on  the time iteration  $k\in\{1, \ldots,  N\}$, 
	$a_{j,k} =a_j
	\left(
		Y^{(1)}_{k},
		\ldots, Y^{(d)}_{k}
	\right)$  and $b_{j,k} =
	b_{j}
	\left(
			Y^{(-j)}_k
	\right)$. We define the \SM method for 
	the multivariate case under 
a split-step  formulation as:
\begin{eqnarray}
 	Y_{k}^{{\star}^{(j)}} &=& Y_k + h\, \varphi_{f^{(j)}}(Y^{\star}_k), \label{eqn:SSLSM1}\\
	Y_{k+1}^{(j)}	&=& Y_k^{{\star}^{(j)}} + g^{(j)}(Y_k^{\star})\, \Delta W_k 
	\label{eqn:SSLSM2},
\end{eqnarray}
with $Y_0=y_0$ and  $	\varphi_{f}(Y_k^{\star})=
		\left(
			\varphi_{f^{(1)}}(Y_k^{\star}),
			\ldots,
			\varphi_{f^{(d)}}(Y_k^{\star})
		\right)$
where
\begin{equation}\label{fi}
	\varphi_{f^{(j)}}\left(Y_k^{\star}\right)
		=
		\left(
			\frac{1}{Y_{k}^{\star(j)}-Y_{k}^{(j)}}
			\int 
			%\limits
				_{Y_{k}^{(j)}}^{Y_{k}^{\star(j)}}
				\frac{du}
				{
					a_{j,k} u
					+b_{j,k}
				}
		\right)^{-1}.
\end{equation}	
The Linear Steklov algorithmic is an implicit scheme, but 
we can derive an explicit matrix version given by  
\begin{equation}\label{m1}
 Y_{k+1} = A^{(1)}(h,Y_k) \,Y_k + A^{(2)}(h,Y_k)b(Y_k)+ g(Y_k)\, \Delta W_k, 
\end{equation}
where  $A^{(1)}= A^{(1)}(h,u)$,  $A^{(2)}=A^{(2)}(h,u)$  and $b(u)=\left(
		b_1(u^{(-1)}), \dots , b_d(u^{(-d)})
	\right)^T$   are defined  by
\begin{align}\label{eqn:SolutionFunctions}	
	A^{(1)}&:=
		\begin{pmatrix}
			e^{ha_1(u)} & \multicolumn{2}{c}{\text{\kern0.5em\smash{\raisebox{-1ex}{\Huge 0}}}} \\
			&\ddots\\
			\multicolumn{2}{c}{\text{\kern-0.5em\smash{\raisebox{0.95ex}{\Huge 0}}}} 
			& e^{ha_d(u)}
		\end{pmatrix},
		\notag
		\\
		%	
	A^{(2)}&:=
	\begin{pmatrix}
		\left(
			\displaystyle
			\frac{e^{ha_1(u)} - 1}{a_1(u)}
		\right)\1{E_1^c}% + h \1{E_i}  & 
		&\BigZero\\
		\qquad \qquad \qquad \qquad\ddots&\\
		\BigZero&
		\left(
			\displaystyle
			\frac{e^{ha_d(u)} - 1}{a_d(u)}
		\right)\1{E_d^c}% + h \1{E_i} 
	\end{pmatrix}
	+h
	\begin{pmatrix}
		\1{E_1} & \BigZero\\
		\qquad \qquad \ddots\\
		\BigZero & 
		\1{E_d}
	\end{pmatrix}.	
\end{align}
It is worth mentioning that the explicit formulation \eqref{m1}-\eqref{eqn:SolutionFunctions} 
is well defined in the set $E_j:=\{x \in \R^d: a_j(x)=0\}$. In the following, we 
assume two important hypotheses which are crucial to prove existence 
of the explicit Linear Steklov method.

\begin{hypothesis}\label[hypothesis]{ass:ajBound} %\label{as:aiFunctions}
	For each component function $f^{(j)}:\R^d:\to \R$, %$j \in \{1, \dots, d \}$,
		$j \in \{1,\dots, d\}$:
	\begin{enumerate}[({A}-1)]
		\item\label{ass:FunctionStructure}
		There are two locally Lipschitz functions $a_j:\mathbb{R}^{d} \to \mathbb{R}$, and
		$b_{j}:\mathbb{R}^{d-1} \to \mathbb{R}$ such that 
		\begin{equation*}\label{eqn:AlternativeConstruction}
		f^{(j)}(y) = a_j (y) y^{(j)} + b_{j}(y^{(-j)}), \qquad
		y^{(-j)} = (y^{(1)}, \dots ,y^{(j-1)}, y^{(j+1)}, \dots, y^{(d)}).
		\end{equation*}
		\item\label{a}
		There is a positive constant $L_a$ such that
		\begin{equation*}
		a_{j}(x) \leq L_a, \qquad \forall x\in \R^d, \quad j=1,\ldots, d.
		\end{equation*}
		\item Each function $b_j(\cdot)$ satisfies the linear growth condition
		\begin{equation*}\label{eqn:bjLinearGrowthCondition}
		|b_j(u^{(-j)})|^2 \leq L_{b}(1+|u|^2) , \qquad \forall x\in \R^d, \quad j=1,\ldots, d.
		\end{equation*}
		\item For each $x\in E_j$ there is an open ball of positive 
		radius and center $x$, $B_r(x)$, such that 
		$$
		\frac{\partial a_j(u)}{\partial u^{(j)}}
		\neq 0, \qquad \forall u \in B_r(x) \setminus E_j.
		$$
	\end{enumerate}
\end{hypothesis}
\begin{hypothesis}\label{ass:HypThmSingularities}
	The set $E_j:=\{x\in \R^{d}: a_j(x)=0\}$ satisfies either:
	\begin{enumerate}[(i)]
		\item  All points of  $ x\in E_j$  which are non isolated zeros of $a_j$ and:
			\begin{itemize}
				\item The set $D:=\{u \in B_r(x): a_j(u)=e^{ha_j(u)}-1= 0\}$ 
					is a smooth curve through $x$. 
				\item
					The canonical vector $e_j$ is not
					tangent to $D$.
				\item
					For each $x \in E_j$, there is an open ball with center
					on $x$ and radio $r$, $B_r(x)$, such that  
					$$
						a_j\neq 0, \qquad
						\frac{\partial a_j(u)}{\partial u^{(j)}} \neq 0 ,\qquad 
						\forall u \in B_r(x)
						\setminus D.
					$$	
				%				
				%				\item
				%					The limit
				%					$$
				%						\lim_{
				%							\substack{
				%								u\to u^*\\ 
				%								u \in B_r(q) \setminus D
				%							}
				%						}
				%						\frac{e^{h a_j(h)} \frac{\partial a_j(u)}{\partial u^{(j)}}}{\frac{\partial a_j(u)}{\partial 
				%						u^{(j)}}}
				%					$$
				%					
			\end{itemize}	
		\item
			All points $x \in E_j$ which are isolated zeros of $a_j$ and:
			\begin{itemize}
				\item
					For each $x\in E_j$,  $x$ is not a limit point of the set 
					$E_{\alpha}:=\{u \in \R^d: (a_j)_\alpha(u)=0\}$.
				\item
					For each $x \in E_j$ there is a star-like set, $x\in E_x$, such that
					the $x$ directional derivative at $u$ satisfies
					$$
						 (a_j)_\alpha(u) \neq 0, \qquad \forall u\in E_x.
					$$
			\end{itemize}		
	\end{enumerate}	
\end{hypothesis}
Next we prove our main theorem, which under the previous assumptions, shows that
the explicit LS method  \eqref{m1}-\eqref{eqn:SolutionFunctions} always exists, 
the function $\varphi$ is bounded by the drift function $f$ and also
the coefficients $\varphi$ and $g$ satisfy a monotone condition. First, we will give a lemma.
\begin{lem}\label{l1}
 Assume \Cref{ass:OSLC,ass:ajBound,ass:HypThmSingularities} hold. The function $\Phi_j(u)=\Phi(h, a_j)(u)$ 
 defined by
 \begin{equation}\label{eqn:ExpBound}
		\Phi_j(u):=\frac{e^{ha_j(u)}-1}{ha_j(u)},
	\end{equation}
	is bounded  on $\R^d$ for each $j\in \{ 1,\dots, d\}$ 
	and we denote its bound by $L_{\Phi}$.
\end{lem}
\begin{pf}    By \Cref{ass:OSLC}, the operator $\Phi$ is continuous
	on $E_j^c$, thus 
	\begin{equation}\label{e0}
		\lim_{h\to 0}
		\frac{e^{ha_j(u)}-1}{ha_j}=1,
	\end{equation}
	for each fixed $u\in E_j^c$. If $u^*\in E_j$ and fixing any $h$, by \Cref{ass:HypThmSingularities},  we obtain 
	one of the following cases:
	\begin{equation}\label{eqn:ajSingularityCasei}
				\lim_{
					\substack{
						u \to u^*\\ 
						u\in E_j^c
					}
				}
				\Phi(h,a_j)(u) =
				%
				\lim_{
					\substack{
						u \to u^*\\ 
						u\in E_j^c
					}
				}	
				\frac{\frac{\partial a_j(u)}{\partial u^{(j)}} 
					h e^{h a_j(u)} 
				}{
					h\frac{\partial a_j(u)}{\partial u^{(j)}}
				}=1,
			\end{equation}	
or
			\begin{equation}\label{eqn:ajSingularityCaseii}
			\lim_{
				\substack{
					u \to u^*\\ 
					u\in E_j^c
				}
			}
			\Phi(h,a_j)(u) 
			=
			%
			\lim_{
				\substack{
					u \to u^*\\ 
					u\in E_j^c
				}
			}	
			\frac{
				\left(
					 e^{h a_j(u)} - 1
				 \right)_{\alpha}
			}{
				\left(
					h a_j(u)
				\right)_{\alpha}
			}	=	1, \qquad \alpha = 0,\pi, 2\pi,\dots
		\end{equation}	
	From \eqref{e0}, \eqref{eqn:ajSingularityCasei} and \eqref{eqn:ajSingularityCaseii} we can deduce that 	
	\begin{equation}\label{eqn:PhiBound}
		\left|\frac{e^{ha_j(u)}-1}{ha_j(u)}
		\right|\leq\left|\frac{e^{hL_a}-1}{ha^*_j}
		\right|,
		\qquad \forall u \in \R^d.
	\end{equation}
	where $a^*_j:= \inf_{u\in E_j^c}\{|a_j(u)|\}$. Even though $a^*_j=0$, we can
	use an argument similar to \crefrange{eqn:ajSingularityCasei}{eqn:ajSingularityCaseii}.	\qed	
\end{pf}

We can  now state the main theorem.

\begin{thm}\label{lem:PhiFhProp}
	Assume \Cref{ass:OSLC,ass:ajBound,ass:HypThmSingularities} hold, and $A^{(1)}$, $A^{(2)}$, $b$  
	defined by 
	\eqref{eqn:SolutionFunctions}. Then given $u\in\mathbb{R}^d$, the equation
	\begin{equation}\label{eqn:varphiEquation}
		v = u + h \varphi_f(v),
	\end{equation}
	has a unique solution 
	\begin{equation}\label{eqn:varphiEqnSolution}
		v = A^{(1)}(h,u)u +A^{(2)}(h,u) b(u)	.
	\end{equation}
%
	If we define the functions
	$F_h(\cdot)$, $\varphi_{f_h}(\cdot)$ and $g_h(\cdot)$ by
	\begin{equation}\label{eqn:FunctionshDefinition}
		F_h(u) = v,
%			\left(
%				u + \frac{b}{a(u)}
%			\right)\exp\left(ha(u)\right)
%				-\frac{b}{a(u)},
			\qquad 
			\varphi_{f_h}(u) =\varphi_{f}(F_h(u)),
			\qquad
			g_h(u) = g(F_h(u)),
	\end{equation}
	then $F_h(\cdot)$, $\varphi_{f_h}(\cdot)$, $g_h(\cdot)$ are local Lipschitz functions 
	and for all $u\in \mathbb{R}^d$ and each $h$ fixed, there is a positive constant $L_{\Phi}$ such that
	\begin{equation}\label{eqn:PhifhFbound}
		|\varphi_{f_h}(u)|\leq L_{\Phi} |f(u)|. 
	\end{equation} 
	Moreover, for each $h$ fixed,
	%$g_h$ is a locally Lipschitz function and 
	 there are positive constants $\alpha^*$ and  $\beta^*$ such that
	\begin{equation}\label{eqn:h-MonotoneCondition}
		\innerprod{\varphi_{f_h}(u)}{u} \vee |g_h(u)|^2 \leq \alpha^* + \beta^* |u|^2, 
		\qquad
		\forall u \in \R^d.
	\end{equation}
\end{thm}
%
\begin{pf}
	%\todo{Justify the uniform convergence of  $\varphi_{f_h}$ and $g_h$.}
	Let us first prove that \eqref{eqn:varphiEqnSolution} is solution 
	of equation \eqref{eqn:varphiEquation}.  Note that 
	\begin{equation}\label{e1}
		v^{(j)} = u^{(j)} + h \,\varphi_{f^{(j)}}(v),	
	\end{equation}
	for each $j\in \{1,\dots, d\}$ and using the linear Steklov function \eqref{fi}, 
	 we can derive that
	\begin{equation}
		v^{(j)}= e^{h a_j(u)} u^{(j)} + 
		\left[
			h\Phi_j(u)
			\1{E_j^c}
			+h \1{E_j}
		\right]b_{j}(u^{(-j)}),
	\end{equation}	
	 which is the $j$-component of the vector
	$A^{(1)}u +A^{(2)} b(u)$. Now we prove inequality \eqref{eqn:PhifhFbound}. Given that 
	$v =\varphi_{f}(F_h(u))$, 
	we can also rewrite \eqref{e1} as
	$$
		\varphi_{f_h}^{(j)}(u) = 
			\frac{
%				\displaystyle
				F_h^{(j)}(u)-u^{(j)}
			}%
			{%				
				\displaystyle
				\int_{u^{(j)}}^{F_h^{(j)}(u)}
				\frac{dz}{a_j(u)z + b_j(u^{(-j)})}
			}.
	$$
	If $u\in E_j$ then $\varphi_{f_h}^{(j)}(u) = b_j(u^{(-j)}) = f(u)$,
	so $L_{\Phi}\geq 1$ fulfills  \eqref{eqn:PhifhFbound}.
	On the other hand, if $u\in E_j^c$ then
	\begin{equation}\label{eqn:VarPhiEjc}
		\varphi_{f_h}^{(j)}(u) =
		\frac{
				(F_h^{(j)}(u)-u^{(j)}) a_j(u)
			}
			{
				\underbrace{
				\ln \left(
					a_j(u) F_h^{(j)}(u) + b_j(u^{(-j)})
				\right)
				}_{:=R_1}
				-
				\ln \left(
					a_j(u) u^{(j)} + b_j(u^{(-j)})
				\right)
			}=\Phi_j(u)f^j(u),		
	\end{equation}
	 where
	\begin{equation}\label{eqn:Ter1Simplification}	
		R_1=
			\ln\left\{
				a_j(u)\left[
					e^{ha_j(u)}u^{(j)} +
					h\Phi_j(u) b_j(u^{(-j)})	
				\right]
				+b_j\left(u^{(-j)}\right)
			\right\} \notag \\
		=h a_j(u) + \ln \left( f^{(j)}(u) \right).		
	\end{equation}
	By lemma \ref{l1}, inequality \eqref{eqn:PhifhFbound} is satisfied 
	for all $u\in E_j \cup E_j^c$. 	As $g_h(x)=g\left(F_h(x) \right)$, 
	by \Cref{ass:OSLC}, then
	\begin{equation} \label{eqn:gLocalLipschitzArg} 
		|g_h(u)-g_h(v)|^2 \leq
			L_g|F_h(u)- F_h(v)|^2  \leq
			2L_g \underbrace{
				|A^{(1)}u - A^{(1)}v|^2 
			}_{:=R_2} +
			2L_g \underbrace{
				|A^{(2)}b(u) - A^{(2)}b(v) |^2 
			}_{:=R_3}.			
	\end{equation}
	Let us consider each term of the right hand of inequality \eqref{eqn:gLocalLipschitzArg}.
	First, note that $A^{(1)}$ is a continuous differentiable function on all 
	$\R^d$, so using the mean value theorem,
	we have
	\begin{equation}\label{eqn:BoundTer1gh}
		R_2
		\leq
		L_{A^{(1)}}|u-v|^2, \quad u,v \in \R^d, \quad |u|\vee |v|\leq n,
	\end{equation}
	for a positive constant $L_{A^{(1)}} =\sup_{0\leq t \leq 1} 
		|\partial A^{(1)}(h, u+t(v-u))|^2$. Meanwhile, 
	\begin{eqnarray}
	R_3 &=&\sum_{j=1}^d\Big[\1{E_j^c}(u) \Phi_j(u) b_j(u^{(-j)}) + h \1{E_j}(u) b_j(u^{(-j)}) 
			-\1{E_j^c}(v) \Phi_j(v) b_j(u^{(-j)})\nonumber \\ &-& h \1{E_j}(v) b_j(v^{(-j)})
			\Big]^2 
			\leq
			4\sum_{j=1}^d
			\Big[
				\left(	
					\1{E_j^c}(u) L_{\Phi} b_j(u^{(-j)})  
				\right)^2
				+
				\left(
					h \1{E_j}(u) b_j(u^{(-j)})
				\right)^2
				\nonumber\\&+&
				\left(
					\1{E_j^c}(v) L_{\Phi} b_j(v^{(-j)})
				\right)^2
				+
				\left(
				 h \1{E_j}(v) b_j(v^{(-j)}
				\right)^2
			\Big].
		\label{eqn:BoundArgTer2}	
	\end{eqnarray}	
	Since  $b^2_j(\cdot)$ is a function of class $C^1(\R^d)$, there is a 
	constant $L_b=L_b(n)$ such that
	\begin{dmath}[label={eqn:Boundbju}]
		|b_j(u)|^2 \leq L_b 
		\condition{
			$\forall u \in \R^d$,
			\quad $|u| \vee |v| \leq n$,}		
	\end{dmath}
	for each $j \in \{1,\cdots, d\}$. Using this bound in  \eqref{eqn:BoundArgTer2}, we obtain
	\begin{equation}\label{eqn:BoundL0Ter2gh}
			R_3\leq
			4 \sum_{j=1}^d
				\left[
					2 L_{\Phi} L_b +2h^2 L_b
				\right] 
			\leq L_0, \qquad
				\forall u,v \in \R^d,
				\quad |u|\vee|v| \leq n,
	\end{equation}
	where $L_0=8d L_b(n)(L_{\Phi}+h^2)$.
	By inequalities \eqref{eqn:BoundTer1gh} and \eqref{eqn:BoundL0Ter2gh},  we get
	\begin{equation}
			|g_h(u) - g_h(v)|^2 \leq L_{g_h}(n) |u - v|^2, 
			\qquad	\forall u,v \in \R^d,
				\quad |u|\vee |v| \leq n,		
	\end{equation}
	 where $L_{g_h}(n)\geq n^2+1+L_0+ L_{A^{(1)}}$. Then $g_h(\cdot)$ is a locally 
	 Lipschitz function. Furthermore, note that under some modifications
	this argument can be used to prove that $F_h(\cdot)$ is also a locally Lipschitz 
	function, which implies that $\varphi_{f_{h}}$ is a locally Lipschitz function.
	Finally, we  will demonstrate inequality \eqref{eqn:h-MonotoneCondition}. 
	By \Cref{ass:OSLC,ass:ajBound}, we have
	\begin{equation*}\label{eqn:bDotProdBound}
			\innerprod{f(u)}{u}
			=
			\sum_{j=1}^d
				a_j(u) \left( u^{(j)} \right)^2
			+
			\sum_{j=1}^d
			b_j(u) u^{(j)}
			\leq \alpha +\beta |u|^2,
	\end{equation*}
	and
		\begin{equation*}\label{eqn:bDotProdBound}
			\innerprod{b(u)}{u} \leq \alpha + (\beta + L_a)|u|^2.
		\end{equation*}
	Using these inequalities and \eqref{eqn:PhifhFbound}, we deduce that
	\begin{equation}\label{eqn:VarphifhDotProdBound}
			\innerprod{\varphi_{f_h}(u)}{u} 
				=\sum_{j=1}^d
						\Phi_j(u) f^{(j)}(u) u^{(j)}
				\leq L_{\Phi} L_a|u| +  L_{\Phi}(\alpha + (L_a + \beta)|u|^2) \leq
				L_{\varphi_{f_h}} (1+|u|^2).					
		\end{equation}
	where
	$
		L_{\varphi_{f_h}}\geq 2 L_{\Phi}\max\{L_a, \alpha, \beta\} + 1.
	$ 
	Meanwhile, since $g$ is globally Lipschitz then
	\begin{equation}\label{eqn:GloballyLipg}
		|g_h(u)|^2 
		\leq
			2 |g(F_h(u)) - g(F_h(0))|^2  + 2 | g(F_h(0))|^2 \leq
			4 L_g |F_h(u)|^2  + 8 L_g |F_h(0)|^2  + 4 |g(0)|^2 .
	\end{equation}
	Now, we bound each term on the right-hand side of  \eqref{eqn:GloballyLipg}.
	By the monotone condition \eqref{eqn:MonotoneCondition}, $|g(0)|^2 \leq 2\alpha$.
	Moreover,
	\begin{equation*}
		|F_h^{(j)} (0) |=
			h\Phi_j(0)\,| b_j(0)| \1{E_j^c}(0) + 
			h\,|b_j(0)| \1{E_j}(0)
			\leq
			\frac{b_0^*}{a_0^*}
			e^{h L_a} (1+h),
			\quad
			\forall j \in \{1, \cdots, d\}.
	\end{equation*}
	where
	$a^*_0 := 
			\min_{
				\substack{
					j \in \{1, \cdots, d \}\\
					a_j(0) \neq 0
				}
			}
			\left\{
				|a_j(0)|
			\right\}$
 and 
		$b^*_0 :=
			\max_{\substack{
				j\in \{1,\cdots, d\}}
			}
			\left\{
				|b_j(0)|
			\right\}.$
	Then
	\begin{equation} \label{eqn:BoundFhZero}
		|F_h(0)|^2 
			\leq
			d\left(
				\frac{b_0^*}{a_0^*}
			\right)^2
		e^{2h L_a} (1+h)^2.
	\end{equation}
	As $\Phi_j$ is bounded, we get
	\begin{dmath*}
		F_h^{(j)}(u) 
			=
			e^{h a_j(u)} u^{(j)} +
			h \Phi_j(u) b_j(u) \1{E_j^c}(u) +h b_j(u) \1{E_j}(u)
			\leq
			e^{h a_j(u)} |u^{(j)}| +
			h L_{\Phi} |b_j(u)| \1{E_j^c}(u) +h |b_j(u)| \1{E_j}(u).
	\end{dmath*}
	 And by \Cref{ass:ajBound},
	\begin{eqnarray}
		|F_h^{(j)}(u)|^2 
		&\leq&
			3 e^{2h L_a }|u|^2 + (3 h^2 L_{\Phi}^2 L_b + 3h^ 2L_b) (1+|u|^2) \nonumber\\
		&\leq&
			3 h^2 L_b (1 + L_{\Phi}^2)   +
			3 \left(
				 e^{2 h L_a} + h^2 L_b (L_{\Phi}^2 + 1 ) 
			\right)|u|^2 \nonumber\\ &\leq& L_F(1+|u|^2), \label{eqn:BoundFhu}
	\end{eqnarray}
	where
	$L_F\geq 3 d \max\{e^{2h L_a},  h^2 L_b(L_{\Phi}^2+1)\}$.
	Using \eqref{eqn:BoundFhZero} and \eqref{eqn:BoundFhu} in  inequality \eqref{eqn:GloballyLipg} yields
	\begin{equation*}
		|g_h(u)|^2 \leq
			4 L_g L_F(1+|u|^2)
			+ 8 L_g d
			\left(
				\frac{b_0^*}{a_0^*}
			\right)^2
			e^{2h L_a} (1+h)^2
			+8 \alpha.
	\end{equation*}
	Therefore, taking
	$
		 L_{g_h} 
		 \geq 
			  4 L_g L_F + 8 L_g d
			  \left(
				  \frac{b_0^*}{a_0^*}
			  \right)^2
			  e^{2h L_a} (1+h)^2
			  +8 \alpha		  
	$
	then
	\begin{equation}\label{eqn:Boundghu}
		|g_h(u)|^2
		\leq
			L_{g_h}(1+|u|^2).		
	\end{equation}
	Hence, from inequalities \eqref{eqn:VarphifhDotProdBound} and \eqref{eqn:Boundghu} 
	and taking for each fixed $h>0$, $\alpha^* := L_{\varphi_{f_h}}\vee L_{g_h}$ and
	$\beta^* := 2\alpha^*$, we obtain inequality  \eqref{eqn:h-MonotoneCondition}.\qed
\end{pf}
%
\begin{remark}\label{rmk:PertrubedSDE}
	It is worth mentioning that if $b_j=0$ in (A-1) 
	then Hypotheses \ref{ass:ajBound} and \ref{ass:HypThmSingularities} are superfluous to 
	prove \Cref{lem:PhiFhProp}. Several  applications  as  stochastic
	Lotka-Volterra systems \cite{Mao2002,Mao2003},  
	the Ginzburg-Landau SDE \cite[Sec. 4.4]{Kloeden1992},
	the damped Langevin Equations where the potential lacks of a constant term 
	(for example $U(x) = \frac{1}{4}|x|^4 - \frac{1}{2}|x|^2$) \cite{Hutzenthaler2012a}.
\end{remark}

This theorem holds the key to demonstrate strong convergence of the explicit LS method. 

