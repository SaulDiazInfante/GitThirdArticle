In this section we remind some classical results  about the moment boundedness, existence 
and uniqueness of the solution of the stochastic differential system \eqref{eqn:SDE1}, 
see \cite{Higham2002b,Mao2013,Mao2007}. Moreover, we state some theorems about the strong 
convergence of the Euler-Maruyama given by Higham et al. in \cite{Higham2002b} which will be useful 
to prove the strong convergence of the Linear Steklov method. Finally, 
we give several definitions and theorems of the multivariable calculus 
that we consider to justify the existence of the Linear Steklov approximation, 
 for references \cite{Lawlor2012,FineAIandKass1966}. Let us assume
the following:


	
\begin{hypothesis}\label[hypothesis]{ass:OSLC}
	The coefficients of SDE \eqref{eqn:SDE1} satisfy the conditions:
	\begin{enumerate}[({H}-1)]
		\item \label{ass:C1Functions}
		The functions $f,g$ are in the class $C^{1}(\R^d)$.
		\item
		\textbf{Local, global Lipschitz condition}. For each integer $n$, there is a positive
		constant $L_{f}=L_{f}(n)$ such that
		$$
		|f(x)-f(y)|^2 %\vee |g(x)-g(y)|^2
		\leq L_{f}|x-y|^2 \qquad \forall x,y \in \R^d, \qquad |x|\vee|y|\leq n,
		$$
		and there is a positive constant $L_g$ such that
		$$
		|g(x)-g(y)|^2 \leq L_{g}|x-y|^2,
		\qquad  \forall x,y \in \R^d.
		$$ 
		\item\label{ass:MonotoneCondition}
		\textbf{Monotone condition.} There exist two positive constants $\alpha$ and $\beta$
		such that
		\begin{equation}\label{eqn:MonotoneCondition}
		\innerprod{x}{f(x)} +\frac{1}{2}|g(x)|^2
		\leq \alpha +\beta |x|^2, \qquad \forall x \in \R^d.
		\end{equation}
	\end{enumerate}
\end{hypothesis}


Under Hypothesis \ref{ass:OSLC} we can assure  existence and uniqueness 
of the solution of continuous system \eqref{eqn:SDE1}  as well as bounds on its moments
 in order to justify the development of a numerical approximation. Next we enumerate the classical 
 results that we mentioned above.
%
\begin{thm}
	Assume \Cref{ass:OSLC}  then for all $y(0)=y_0\in \mathbb{R}^d$   there exists a 
	unique global solution $\{y(t)\}_{t\geq 0}$ to SDE \eqref{eqn:SDE1}. Moreover, the solution has the 
	following properties for any $T>0$,
	\begin{equation*}
		\ms{y(T)}< 
		\left(
			|y_0|^2 +2\alpha T 
		\right)e^{2\beta T},
	\end{equation*}
	and
	\begin{equation*}
	\Prob{\tau_n\leq T}
	\leq \frac{
		\left(
		|y_0|^2 +2\alpha T 
		\right)
		e^{2\beta T}
	}{n},
	\end{equation*}
	where $n$ is any positive integer and 
	%\begin{equation*}
	$\tau_n := \inf \{ t\geq 0 : |y(t)|>n\}$.
	%\end{equation*}
\end{thm}
%
\begin{thm}
	\label{thm:MaoCoercive}
	Let $p\geq 2$ and $x_0\in L^p(\Omega, \mathbb{R}^d)$. Assume that there exits a constant $C>0$
	such that for all $(x,t)\in \mathbb{R}^d\times [t_0,T]$,
	\begin{equation*}
	\innerprod{x}{f(x,t)}+\frac{p-1}{2}|g(x,t)|^2 \leq C(1+|x|^2).
	\end{equation*}
	Then
	\begin{equation*}
	\m|y(t)|^p
	\leq
	2^{\frac{p-2}{2}}
	\left(
	1 + \m|y_0|^p
	\right)e^{Cpt} \quad \text{ for all } t\in[0,T].
	\end{equation*}
\end{thm}
%
\begin{lem}
	\label{lem:MomentBound}
	Assuming \Cref{ass:OSLC}, for each $p\geq 2$, there is a $C=C(p,T)$ such that
	\begin{equation*}
	\EX{\sup_{0\leq t \leq T}|y(t)|^p}\leq C \left(1+\mep{y_0}\right).
	\end{equation*}
\end{lem}

Now we  state some  results of the multivariable calculus  like  
the L'H\^{o}pital Rule and the existence  of the directional derivative 
at an isolated singularity that will be used  throughout the paper.

\begin{dfn}
	Let $u,\mathbf{q}\in \R^2$ and $\alpha$ the positive angle formed by the $x$-axis and the segment
	$\overline{u \mathbf{q}}$.	We denote by 
	\begin{align*}
		f_{\alpha}(u) &= 
			\cos(\alpha) 		
			\frac{\partial f}{\partial u^{(1)}}(u) + 
			\sin(\alpha)
			\frac{\partial f}{\partial u^{(2)}}(u) 
			= \frac{ \innerprod{q-u}{\nabla f(u)}}{|u-q|},			
	\end{align*}
	the $\mathbf{q}$ \emph{directional derivative at $u$}.
\end{dfn}
\begin{dfn}
	A set $S\subset \R^2$ is \emph{star-like} with respect to a point $\mathbf{q}$, if for each point $s \in S$ the open 
	segment $\overline{s \mathbf{q}}$ is in $S$.
\end{dfn}

\begin{thm}\label{thm:Lawlor}
	Let $\mathcal{N}$ be a neighborhood in $\R^2$ of a point $\mathbf{q}$ for  which
	the two differentiable functions $f:\mathcal{N}\to \R$ and $g:\mathcal{N}\to \R$. Set 
	$$
		C=\{x \in \mathcal{N}: f(x)=g(x)=0 \},
	$$
	assuming that $C$ is a smooth curve through $\mathbf{q}$ and 	
	there exist a vector $\mathbf{v}$ not tangent to $C$ at $\mathbf{q}$
	such that the directional derivative $D_{\mathbf{v}}g$ of $g$ in the direction of $\mathbf{v}$ 
	is never zero within $\mathcal{N}$ and $\mathbf{q}$ is a limit point of $\mathcal{N}\setminus C$. Then
	\begin{equation*}
		\lim_{(x,y)\to \mathbf{q}}
		\frac{f(x,y)}{g(x,y)} =
		\lim_{
				\substack{
					(x,y)\to \mathbf{q}\\ 
					(x,y)\in \mathcal{N} \setminus C
				}
		}
		\frac{D_{\mathbf{v}} f }{D_{\mathbf{v}} g},
	\end{equation*}
	if the latter limit exists.
\end{thm}



\begin{thm}\label{thm:Fine}
	Let $\mathbf{q}\in \R^2$ and let $f$,$g$ be functions whose domains include a set $S\subset \R^2$ which is 
	star-like 
	with  respect to the point $\mathbf{q}$. Suppose that on $S$ the functions are differentiable and that
	the directional derivative of $g$ with respect to $\mathbf{q}$ is never zero. With the understanding that all 
	limits are taken from within on $S$ at $\mathbf{q}$ and if
	$$f(\mathbf{q})=g(\mathbf{q})=0 \qquad \mbox{and} \qquad
				\displaystyle
				\lim_{x \to \mathbf{q}}
				\frac{f_{\alpha}(x)}{g_{\alpha}(x)} = L,$$
	then
	$$
		\lim_{x \to \mathbf{q}}
		\frac{f(x)}{g(x)} = L.
	$$
\end{thm}
 Finally, the next theorems gives the necessary conditions to assure strong convergence and 
 the convergence rate of the Euler-Maruyama (EM) 
 method.  
\begin{thm}\label{thm:HighamMaoStuart}
	Assume \Cref{ass:OSLC} holds, then the EM scheme given by: 
	\begin{equation}\label{eqn:EulerMaruyamaHigham}
	Y^{EM}_{k+1}= Y^{EM}_k+hf(Y^{EM}_k) + g(Y^{EM}_k)\Delta W_k,
    \end{equation} 
	where $h$ is the step-size and its
	continuous  extension
	\begin{equation}\label{extE}
	\overline{Y}^{EM}(t):=Y_0+\int_0^t f (Y^{EM}_{\eta(s)})\,ds+\int_0^t g(Y^{EM}_{\eta(s)})\,dW(s),
	\end{equation}
	where $\eta(t):=k$ for $t\in[t_k,t_{k+1})$, satisfies
	\begin{equation}
		\lim_{h\to 0}
		\EX{\sup_{0\leq t\leq T}|\overline{Y}^{EM}(t)-y(t)|^2}=0.
	\end{equation}
\end{thm}	
Under the following assumptions, we can get the rate of convergence of the EM scheme.
\begin{hypothesis}\label{PolynomialGrowth}
	There exist constants $L_f, D\in \R$ and $q \in \Z^+$ such that $\forall u,v \in \R^d$
	\begin{eqnarray}
		\innerprod{u-v}{f(u)-f(v)}
			&\leq& L_f|u-v|^2, \nonumber \\
		|f(u) - f(v)|^2 
			&\leq& 
				D(1 + |u|^q +|v|^q) |u-v|^2.\nonumber
	\end{eqnarray}
\end{hypothesis}

\begin{hypothesis}\label{ass:MomentBounds}
	The SDE \eqref{eqn:SDE1} and the EM solution \eqref{eqn:EulerMaruyamaHigham} satisfies
	\begin{equation*}
		\EX{
			\sup_{0\leq t\leq T}
			|y(t)|^p	
		}, \quad
	\EX{
		\sup_{0\leq t\leq T}
			|Y^{EM}(t)|^p	
	}< \infty, \quad
	\EX{
		\sup_{0\leq t\leq T}
		|\overline{Y}^{EM}(t)|^p	
	} < \infty, \qquad \forall p\geq 1.
	\end{equation*}
\end{hypothesis}
\begin{thm}
	Under Hypotheses  \ref{PolynomialGrowth} and \ref{ass:MomentBounds} the EM solution with 
	continuous extension \eqref{extE}
	satisfies
	\begin{equation}
		\EX{
			\sup_{0\leq t \leq T}
			|\overline{Y}^{EM}(t) - y(t)|^2
		} = \mathcal{O}(h^2).
	\end{equation}
\end{thm}

%********************************************************************************************
%              Section 3
%********************************************************************************************

