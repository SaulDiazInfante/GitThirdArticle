
 To prove strong convergence of the Linear Steklov method \eqref{m1}-\eqref{eqn:SolutionFunctions} for  
SDEs with locally Lipschitz coefficients, we rely on the Higham-Mao-Stuart (HMS) technique  
 \cite{Higham2002b}. Recently, several works have used  this  procedure  
 to establish strong convergence for some particular scheme among others \cite{Beyn2010,Guo2014,Hutzenthaler2015,Hutzenthaler2012a,Hutzenthaler2010,Lamba2007,
 Mao2013,Tretyakov2013}. The proof is rather technical and can be divided into the following steps:
 
 \begin{enumerate}[\bf{Step} 1:]
 	\item The explicit LS method \eqref{m1}-\eqref{eqn:SolutionFunctions} is equivalent to 
 	the Euler-Maruyama method \eqref{eqn:EulerMaruyamaHigham} 
	for the conveniently modified SDE
	\begin{equation} \label{eqn:PerturbedHighamSDE}
		dy_h(t)= \varphi_{f_h}(y_h(t))dt +g_h(y_h(t))dW(t),
		\qquad y_h(0)=y_0,  \qquad t\in [0,T],
	\end{equation}
	which is a perturbation of SDE \eqref{eqn:SDE1} given that $|f(x)-\varphi_{f_h}(x)|=\calO(h)$.
	\item\label{stp:PerturbedSolution}
			The solution of the modified SDE \eqref{eqn:PerturbedHighamSDE} has 
			bounded moments and it is 
			"close" to  the solution of the SDEs \eqref{eqn:SDE1}  in the sense of the uniform mean square norm 
			$
				\EX{\sup_{0\leq t\leq T}|\cdot|^2}
			$.
	\item
	\label{stp:MethodBoundedMoments}
		The explicit LS method  has bounded moments.
	\item
		There is a continuous extension of the explicit LS method of the form:
		$$\overline{Y}(t):=Y_0+\int_0^t\varphi_{f_h}(Y_{\eta(s)})\,ds+\int_0^t g(Y_{\eta(s)})\,dW(s), \qquad  
		\eta(t):=k \,\mbox{ for } t\in[t_k,t_{k+1}),$$
		 which has bounded moments.
	\end{enumerate}	
	Proving the above statements and using theorem \ref{thm:HighamMaoStuart},
	we  conclude the strong convergence of the explicit LS method. 
	Step 1 is formalized in the following corollary.
	
%======================================================================================================================
%                                                 STEP 1                                                              %
%======================================================================================================================

\begin{corollary}\label{col:SSSMeEMmod}
	Assume \Cref{ass:OSLC,ass:ajBound,ass:HypThmSingularities} hold, then the 
	explicit \SM method \eqref{m1}-\eqref{eqn:SolutionFunctions} for SDE \eqref{eqn:SDE1} is 
	equivalent to the Euler-Maruyama scheme applied to the modified SDE \eqref{eqn:PerturbedHighamSDE}.
\end{corollary}
\begin{pf}
	Using the functions $\varphi_{f_h}(\cdot)$ and $g_h(\cdot)$ defined in \eqref{eqn:FunctionshDefinition}, 
	we can rewrite the explicit \SM method  as 
	$$
		Y_{k+1} = Y_k + h \varphi_{f_h}(Y_k) + g_h(Y_k)\Delta W_k,
	$$
	which is the EM approximation for the modified SDE \eqref{eqn:PerturbedHighamSDE}. \qed
\end{pf}
%======================================================================================================================
%                                                 STEP 2                                                              %
%======================================================================================================================

	Now we proceed with step 2. In what follows $C$ denotes a generic positive constant independent of the step $h$, and
which value could change at each appearances.
\begin{lem}\label{lem:BoundAndConvergenceOfyh}
	Assume \Cref{ass:OSLC,ass:ajBound,ass:HypThmSingularities} hold, then there is a constant $C=C(p,T)>0$ and a sufficiently small
	step size $h$, such that for all $p>2$
	\begin{equation}\label{eqn:yh-MomentBounds}
		\m\left[
			\sup_{0\leq t \leq T}
				|y_h(t)|^p
		\right]
		\leq
			C
		\left( 
			1+\m |y_0|^p
		\right).
	\end{equation}
	Moreover
	\begin{equation}\label{eqn:yh-convergence}
	\lim_{h \to 0}
	\m\left[
	\sup_{0\leq t \leq T}
	|y(t)-y_h(t)|^2
	\right]=0.
	\end{equation}
\end{lem}
\begin{pf}
	By theorem \ref{thm:MaoCoercive} and inequality \eqref{eqn:h-MonotoneCondition}, 
	we have	 bound \eqref{eqn:yh-MomentBounds}.
	On the other hand, to prove \eqref{eqn:yh-convergence} we will use the properties of 
	$\varphi_{f_h}$ and the Higham's stopping time technique employed in \cite[Thm 2.2]{Higham2002b}. 
	Note that by relationship \eqref{eqn:VarPhiEjc}, we can rewrite the Steklov function  as
	\begin{align}\label{fo}
		\varphi_{f_h}(u) 
			&=  f^{(j)}(u) \1{E_j}(u) +\Phi_j(u) f^{(j)}(u) \1{E_j^c}(u) .
	\end{align}
	 Let $u\in \R^d$, $|u|<n$ by \Cref{l1} we have $|\Phi(u)|\leq L_{\Phi}$ and since $f \in C^1(\R^d)$, there is a 
	 positive  constant $R_n$, which depends on $n$, such that 
	\begin{align*}	
		|\varphi_{f_h}^{(j)}(u) - f^{(j)}(u)|
		&\leq
			\1{E_j^c}(u)
			|f^{(j)}(u)|
			\left|
				\Phi_j(u) - 1
			\right| \notag \\
		&\leq
			\1{E_j^c}(u)
			\left(
				L_{\Phi} + 1
			\right)
			|f(u)|	 \notag \\
		&\leq
		\1{E_j^c}(u) R_n(L_{\Phi}+1), \quad \forall u \in \R^d, \quad |u|\leq n,
	\end{align*}
	for each $j\in \{1,\dots, d\}$. At the interface betweeen $E_j$ and $E_j^c$, 
	we know by the scalar L' H\^{o}pital theorem that 
	$ \lim_{h\to 0} \Phi_j(u) = 1$ for each fixed $u\in E_j^c$.
	And if $u\in E_j$,  by \Cref{ass:HypThmSingularities} and using   
	theorems \ref{thm:Lawlor} or \ref{thm:Fine}, we have 
	\begin{equation*}
	\lim_{h \to 0} F_h^{(j)}(u)
		=
		\lim_{h \to 0}
			e^{ha_j(u)} u^{(j)} + 
		\lim_{h \to 0}
			\left( \Phi_j(u)
				\1{E_j^c}(u)
				+h \1{E_j}(u)
			\right)
			b_j(u^{(j)}) 
		= u^{(j)},
	\end{equation*}
	for each $j \in \{1, \dots , d\}$, hence
	$%\begin{equation*}
		\displaystyle
		\lim_{h\to 0} F_h(u)=u.
	$ %\end{equation*}
	In addition, $g_h$ and $g$ are locally Lipschitz. So, given $n>0$ there is  a function 
	$K_n(\cdot):(0,\infty)\to (0,\infty)$ such that
	$K_n(h)\to 0$ when $h \to 0$ and
	\begin{equation}\label{eqn:PhihGhKRhBound}
		|\varphi_{f_h}(u)-f(u)|^2 \vee |g_h(u)-g(u)|^2
		\leq K_n(h) \qquad \forall u\in \R^d, \quad |u| \leq n.
	\end{equation}
	Also, $f$, $g$ are $C^{1}$, then there is  a constant $H_n>0$ such that
	\begin{equation}\label{eqn:f-gHRBound}
		|f(u)-f(v)|^2 \vee |g(u)-g(v)|^2
		\leq H_n |u-v|^2\qquad \forall u,v \in \R^d, \quad|u|\vee |v| \leq n.
	\end{equation}
	The rest of the proof run as in \citep[Lem. 3.6]{Higham2002b}. \qed
\end{pf}

%======================================================================================================================
%                                                 STEP 3                                                              %
%======================================================================================================================
In the following, we proceed with step 3 which establishes that \SM method has bounded moments. 
This is  the key step in the HSM technique.


\begin{lem}\label{lem:SSSMMomentBounds}
	Assume \Cref{ass:OSLC,ass:ajBound,ass:HypThmSingularities} hold, then for each $p\geq 2$ there is 
	a universal positive constant 
	$C=C(p,T)$ 
	such that for the explicit \SM method \eqref{m1}-\eqref{eqn:SolutionFunctions} is satisfied
	\begin{equation*}
		\m\left[
		\sup_{kh \in [0,T]}
		|Y_k|^{2p}
		\right]\leq C.
	\end{equation*}
\end{lem}
\begin{pf}
	Using a split formulation of the \SM scheme \eqref{m1}-\eqref{eqn:SolutionFunctions} as follows:
	\begin{eqnarray*}
 	Y_{k}^{{\star}^{(j)}} &=& A^{(1)}(h,Y_k) Y_k + A^{(2)}(h,Y_k) b(Y_k), \label{split1}\\
	Y_{k+1}^{(j)}&=& Y_k^{{\star}^{(j)}} + g^{(j)}(Y_k^{\star})\, \Delta W_k 
	\label{split2},
\end{eqnarray*}
	from the first step of this split scheme, using (A-3) and 
	the Cauchy-Schwartz inequality, we get
	\begin{eqnarray}
		|Y_k^{\star}|^{2}
		&\leq&
			|A^{(1)}(h,Y_k)|^2 |Y_k|^2  
			+ 2 \innerprod{A^{(1)}(h,Y_k)Y_k}{A^{(2)}(h,Y_k) Y_k b(Y_k)}
			+|A^{(2)}(h,Y_k)|^2 |b(Y_k)|^2\nonumber\\
		&\leq&
			|A^{(1)}(h,Y_k)|^2 |Y_k|^2  
		+ 2 \sqrt{L_b} d|A^{(1)}(h,Y_k)||A^{(2)}(h,Y_k)||Y_k|(1+|Y_k|)
		\label{leqn:Yn2Bound} 
		\\& & +L_b|A^{(2)}(h,Y_k)|^2 (1+|(Y_k)|^2). \nonumber
	\end{eqnarray}
	From (A-2), we can deduce that
	\begin{dmath}[label=eqn:A1Bound]
		|A^{(1)}(h,Y_k)|^2 
		=
			\left|
				\diag
				\left(
					e^{ha_1(Y_k)}, \dots, e^{ha_d(Y_k)} 
				\right)
			\right|^2
		\leq L_{A^{(1)}},		
	\end{dmath}
	where $L_{A^{(1)}}=d\, e^{ 2 T L_a}$ and also by \eqref{eqn:PhiBound}, we deduce that
	\begin{align}
		|A^{(2)}(h,Y_k)|^2 
		&=
		\left|
			h 
			\diag
			\left(
				\1{E_1}(Y_k)
				+\1{E_1^c}(Y_k)\Phi_1(Y_k), 
				\dots,
				\1{E_d}(Y_k)
				+\1{E_d^c}(Y_k) \Phi_d(Y_k)
			\right)
		\right|^2 \notag \\
		%
		&\leq
		\sum_{j=1}^{d}
		\left(
			\1{E_j^c}
			|h\Phi_j(Y_k)|^2
			+ h^2
		\right)
		\leq
		2 e^{2 L_a  T}
				\sum_{j=1}^d
				\frac{1}{a_j^*} + d T^2\leq L_{A^{(2)}}.
		\label{eqn:A2Bound}
	\end{align}
	Combining \eqref{eqn:A1Bound} and \eqref{eqn:A2Bound}  on  inequality \eqref{leqn:Yn2Bound} yields
	\begin{eqnarray}\label{eqn:YkStarBound}
		|Y_k^{\star}|^2
		&\leq&	L_{A^{(1)}} |Y_k|^2
			+ 2 d \sqrt{L_{A^{(1)}} L_{A^{(2)}} L_b }\,|Y_k|(1+|Y_k|)
			+L_{A^{(2)}} L_b (1+|Y_k|^2) \nonumber\\
			&\leq& 
				C(3|Y_k|^2+|Y_k| + 1) 			
			\leq C(1+|Y_k|^2),
	\end{eqnarray}
	where $C\geq L_{A^{(1)}}+ 2 d \sqrt{L_{A^{(1)}} L_{A^{(2)}} L_b} + 
	L_{A^{(2)}} L_b$. Applying  the bound \eqref{eqn:YkStarBound} 
	in the  second step of the split scheme, we get
	\begin{equation*}
		|Y_{k+1}|^2
		\leq
			C \left(
				|Y_k|^2 + 1
			\right)
			+ 2\innerprod{Y^{\star}_k}{g(Y^{\star}_k) \Delta W_k}
			+ \left|g(Y^{\star}_k) \Delta W_k \right|^2.
	\end{equation*}
	Now, we choose two integers $N,M$ such that $Nh\leq Mh \leq T$. So, adding 
	backwards we obtain
	\begin{equation*}
		|Y_N|^2
			\leq
			S_N\left(
				\sum_{j=0}^{N-1}
					(1+|Y_j|^2)
				+
				2\sum_{j=0}^{N-1}
					\innerprod{Y_j^{\star}}{g(Y_j^{\star}) \Delta W_j}
				+
				\sum_{j=0}^{N-1}
					\left|
						g(Y_j^{\star}) \Delta W_j
					\right|^2
			\right),
	\end{equation*}
	where	$S_N:=	\sum_{j=0}^{N-1}C^{N-j}$. Raising both sides to the power $p$ and 
	using the standard inequality, we obtain
	\begin{align}\label{eqn:RelationToBound}
		|Y_N|^{2p}	
			&\leq
				6^{p} S_N^p
				\left(
					N^{p-1}
						\sum_{j=0}^{N-1}
						(1+|Y_j|^{2p})	
					+
					\left|
						\sum_{j=0}^{N-1}
						\innerprod{Y_j^{\star}}{g(Y_j^{\star}) \Delta W_j}
					\right|^p
					+
					N^{p-1}
					\sum_{j=0}^{N-1}
						\left|
							g(Y_j^{\star}) \Delta W_j
						\right|^{2 p}				
				\right).
	\end{align}
	Now we will show that the second and third terms of inequality \eqref{eqn:RelationToBound} are bounded.
	We denote by $C=C(p,T)$ a generic positive constant which does not depend on  the step size $h$ and whose
	value may change between occurrences.	Next, applying the Bunkholder-Davis-Gundy 
	inequality, we see that
	\begin{align}\label{eqn:BoundSecondTerm}
		\m
		\left[
			\sup_{0\leq N \leq M}
			\left|
				%\exp(2hpNL)
				\sum_{j=0}^{N-1}
					\innerprod{Y_j^{\star}}{g(Y_j^{\star})\Delta W_j}
			\right|^{p}
		\right]
		&\leq
			C\m
			\left[
				\sum_{j=0}^{N-1}
					|Y_j^{\star}|^2
					|g(Y_j^{\star})|^2
					h
			\right]^{p/2}
			\notag\\
		&\leq
			C h^{p/2}M^{p/2-1}
			\m
				\sum_{j=0}^{M-1}
					|Y_j^{\star}|^p (\alpha +\beta |Y_j^{\star}|^2)^{p/2}
			\notag\\
		&\leq
			2^{p/2-1}C T^{p/2-1} h  
			\m
			\sum_{j=0}^{M-1}
				(\alpha^{p/2}|Y_j^{\star}|^p +\beta^{p/2} |Y_j^{\star}|^{2p})
			\notag\\
		&\leq
			C h
			\m
			\sum_{j=0}^{M-1}
				(1+2|Y_j^{\star}|^p + |Y_j^{\star}|^{2p})
			\notag\\
		&\leq
			C h 
			\sum_{j=0}^{M-1}
			\left[
				1+\m|Y_j^{\star}|^{2p}
			\right]
			\notag\\
		&\leq
			C 
			+ 
			C h 
			\sum_{j=0}^{M-1}
				\m|Y_j|^{2p},				
	\end{align}
	 Now, using the Cauchy-Schwartz inequality, the monotone condition 
	\eqref{eqn:MonotoneCondition} and the bound \eqref{eqn:YkStarBound}, we obtain
	\begin{align}
	\m\left[
		\sup_{0\leq N \leq M}
			\sum_{j=0}^{N-1}
			\left|
				g(Y_j^{\star})\Delta W_j
			\right|^{2p}	
		\right]
		&=
			\m
				\sum_{j=0}^{M-1}
					\left|
						g(Y_j^{\star})\Delta W_j
					\right|^{2p}
			\notag\\
		&\leq	
			\sum_{j=0}^{M-1}
				\m
					\left|
						g(Y_j^{\star})
					\right|^{2p}
				\m
					\left|
						\Delta W_j
					\right|^{2p}
			\notag \\
		&\leq
			C h^p
			\sum_{j=0}^{M-1}
			\m
				\left[
					\alpha +\beta|Y^{\star}_j|^2
				\right]^p
			\notag\\
	%
		&\leq
			Ch^p
			\sum_{j=0}^{M-1}
			\m
				\left[
					\alpha ^p +\beta^p |Y^{\star}_j|^{2p}
				\right]
			\notag\\
		&\leq
		Ch^{p-1}
		+
		Ch^p \sum_{j=0}^{M-1}
			\m|Y_j|^{2p} \label{eqn:BoundThirdTerm}.
	\end{align}
	Thus, combining bounds \eqref{eqn:BoundSecondTerm} and \eqref{eqn:BoundThirdTerm} with the inequality 
	\eqref{eqn:RelationToBound}, we can assert that
	\begin{align}
		\EX{
			\sup_{0\leq N \leq M}
					|Y_N|^{2p} 
		}
% 		\leq
% 			C(M,T) + C(M,T)(1+h)
% 			\sum_{j=0}^{M-1}
% 				\m|Y_j|^{2p}
		\leq	
			C +C(1+h) 
			\sum_{j=0}^{M-1}
				\EX{
					\sup_{0\leq N \leq j}
					|Y_N|^{2p}
				}	
		.
	\end{align}
	Finally, using the discrete-type Gronwall inequality, we conclude that
	\begin{align*}
		\EX{
			\sup_{0\leq N \leq M}
			|Y_N|^{2p} 
		}	
		&\leq
			C e^{C(1+h)M} 
		\leq 
		C e^{C(1+T)}<C,
	\end{align*}
	since the constant C does not depend on $h$, the proof is complete. \qed
\end{pf}
%======================================================================================================================
%                                                 STEP 4                                                              %
%======================================================================================================================
	
	
Let $\{Y_k\}$ denote the explicit \SM solution of SDE \eqref{eqn:SDE1} defined by 
the recurrence equations \eqref{m1}-\eqref{eqn:SolutionFunctions}.
By corollary \ref{col:SSSMeEMmod}, we can give a continuous extension for the \SM approximation 
from the time continuous EM extension  \eqref{extE}. 
Moreover, we will also prove in the following theorem that the  moments 
of the continuous LS extension remains bounded.
\begin{corollary}\label{col:ContinuousExtBoundedMoments}
	Assume \Cref{ass:OSLC,ass:ajBound,ass:HypThmSingularities} hold and suppose  $0<h<1$ and $p\geq 
	2$. Then there is a continuous extension $\overline{Y}(t)$ of the 
	Linear Steklov recurrence $\{Y_k\}$ defined by  
	\eqref{m1}-\eqref{eqn:SolutionFunctions}  and a positive constant $C=C(T,p)$ such 
	that
	\begin{equation*}
		\EX{\sup_{0\leq t \leq T} |\overline{Y}(t)|^{2p} }
		\leq C.
	\end{equation*}
\end{corollary}
	\begin{pf}
		We take $t=s+t_k$ in $ [0,T]$ such that $\Delta W_k(s):= W(t_k+s)- W(t_k)$ for $0\leq s <h$.
		Then we define 
		\begin{equation}\label{eqn:SSLSContinuousExtension}
			\overline{Y}(t_k+s):= Y_k + s \varphi_{f_h}(Y_k) + g_h(Y_k)\Delta W_k(s),
		\end{equation}
		as a continuous extension of the \SM scheme \eqref{m1}-\eqref{eqn:SolutionFunctions}. 
		We proceed to show that $\overline{Y}(t)$ has bounded moments.
		Using the split formulation, we have $Y_k^{\star}= Y_k + h \varphi_{f_h}(Y_k)$,
		that is
		\begin{align*}
			Y_k + s \varphi_{f_h}(Y_k)=\gamma Y_k^{\star} + (1-\gamma)Y_k,
		\end{align*}
		where $\gamma = s/h$. Hence, we can rewrite the continuous extension \eqref{eqn:SSLSContinuousExtension} as
		\begin{equation}\label{re1}
			\overline{Y}(t) =
			\gamma Y^{\star}_k + (1-\gamma) Y_k +g_h(Y_k) \Delta W_k(s). %\label{eqn:SSLSMConExt}
		\end{equation}
		Using bound \eqref{eqn:YkStarBound} in \eqref{re1}, we get
		\begin{align}
			|\overline{Y}(t_k+s) |^2 
			\leq \hspace{-0.1cm}
				3\left[
					\gamma C
					+
					\left(
						\gamma C +1 - \gamma
					\right)
					|Y_k|^2
					+
					|g_h(Y_k)\Delta W_k(s)|^2
			\right] \hspace{-0.15cm}\leq\hspace{-0.1cm}
			C
			+
			C
			\left(
				|Y_k|^2 + |g_h(Y_k)\Delta W_k(s)|^2
			\right).
		\notag
		\end{align}
	\end{pf}
	Thus, 
	\begin{equation}
		\hspace{-0.2cm}\sup_{0\leq t\leq T} |\overline{Y}(t)|^{2p}
		\leq
			\hspace{-0.2cm}\sup_{0\leq kh\leq T}
			\left[
				\sup_{0\leq s\leq h}
					|\overline{Y}(t_k+s)|^{2p} 
			\right] 
		\leq
			\hspace{-0.2cm}\sup_{0\leq kh\leq T} 
			\left[
				\sup_{0\leq s\leq h}
					C 
					\left(
						1 + |Y_k|^{2p} + |g_h(Y_k)\Delta W_k(s)|^{2p}
					\right)
			\right],
		\label{eqn:BeforeDoob}
	\end{equation}
	for $t\in [0,T]$.
	%
	Now, taking a non negative integer $0 \leq k \leq N$ such that $0\leq Nh \leq T$, we obtain
	\begin{align}
		\sup_{0\leq t\leq T} |\overline{Y}(t)|^{2p}
		&\leq 
			C
			\left(
				1
				+
				\sup_{0\leq kh\leq T} 
					|Y_k|^{2p}
					+
					\sup_{0\leq s\leq h}
						\sum_{j=0}^N
							|g_h(Y_j)\Delta W_j(s)|^{2p}
			\right) \label{eqn:SumDiffusion}.
	\end{align}
	So, using the Doob's Martingale inequality, lemma
	\ref{lem:SSSMMomentBounds} and that $g_h$ is a locally 
	Lipschitz function, we can bound each term of the inequality \eqref{eqn:SumDiffusion}
	as follows
	\begin{align}
		\EX{
			\sup_{0 \leq s \leq h} |g(Y_j) \Delta W_j(s)|^{2p}
		}
		\leq
			\left(
				\frac{2p}{2p-1}
			\right)^{2p}
			\m|g_h(Y_j)\Delta W_j(h)|^{2p}
% 		&\leq
% 			C \m |g_h(Y_j)|^{2p} \m |\Delta W_j(h)|^{2p} \notag\\
		\leq
			C h^p
			\left(
				1 + \m|Y_j|^{2p}
			\right)
		 \leq C h, \label{eqn:BeforeConclusion}
	\end{align}
	for each $j \in \{0,\dots, N\}$.
	Since $Nh\leq T$ and combining bounds \eqref{eqn:SumDiffusion} 
	and \eqref{eqn:BeforeConclusion}, we conclude the proof. \qed
%======================================================================================================================
%                                                 STEP 5                                                              %
%======================================================================================================================

	We can now state  the strong convergence theorem  for the  explicit Linear Steklov method.%
%We are now in a position to execute the \textbf{Step 5}.
\begin{thm}
	Assume \Cref{ass:OSLC,ass:ajBound,ass:HypThmSingularities} hold, consider the 
	explicit \SM method \eqref{m1}-\eqref{eqn:SolutionFunctions} for SDE \eqref{eqn:SDE1}.
	Then there exist a continuous-time extension $\overline{Y}(t)$ of the LS solution
	 $\{Y_k\}$ for which $\overline{Y}(t)=Y_k$ and
	\begin{equation*}
		\lim_{h\rightarrow0}\Big[\sup_{0\leq t\leq T} |\overline{Y}(t) - y(t)|^2\Big]=0.
	\end{equation*}
\end{thm}
\begin{pf}
	First, we bound  the mean square error of the Steklov approximation as follows
	\begin{align}\label{eqn:AfterTriangle}
		\EX{\sup_{0\leq t \leq T}|\overline{Y}(t) - y(t)|^2}
		&\leq
		2\EX{
			\sup_{0\leq t \leq T}
			|\overline{Y}(t) - y_h(t)|^2
		}
		+
		2\EX{
			\sup_{0\leq t \leq T}
			|y_h(t) - y(t)|^2
		}.
	\end{align}
	 By lemma \ref{lem:BoundAndConvergenceOfyh}, the solution of the modified SDE \eqref{eqn:PerturbedHighamSDE}, $y_h$, has
	$p$-bounded moments ($p\geq 2$), and by corollary \ref{col:ContinuousExtBoundedMoments}, 
	the \SM continuous extension for the SDE \eqref{eqn:SDE1}, $\overline{Y}(t)$, 
	has bounded moments and it is equivalent to the EM extension for the modified SDE 
	\eqref{eqn:PerturbedHighamSDE}. Hence, we can apply 
	theorem	\ref{thm:HighamMaoStuart} to conclude that
	\begin{equation}\label{eqn:FirstTermZeroLim}
		\lim_{h\to 0}
		\EX{
			\sup_{0\leq t \leq T}
			|\overline{Y}(t) - y_h(t)|^2
		} = 0,
	\end{equation}
	 meanwhile by lemma \ref{lem:BoundAndConvergenceOfyh},
	\begin{equation}\label{eqn:SeconTermZeroLim}
		\lim_{h\to 0}
		\EX{
			\sup_{0\leq t \leq T}
			|y_h(t) - y(t)|^2
		} = 0.
	\end{equation}
	Combining limits \eqref{eqn:FirstTermZeroLim} and \eqref{eqn:SeconTermZeroLim} with 
	inequality \eqref{eqn:AfterTriangle}, we have the desired conclusion. \qed
\end{pf}
%
%********************************************************************************************
%              Section 6
%********************************************************************************************

